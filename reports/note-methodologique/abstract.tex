%----------------------------------------------------------------------------------------
%	ABSTRACT
%----------------------------------------------------------------------------------------
% 
\lettrineabstract{Ce document a pour objectif de décrire la méthodologie d'entraînement du modèle
ainsi que la fonction de coût, l'algorithme d'optimisation et la métrique d'évaluation. 
L'interprétabilité du modèle et les limites de ce dernier seront également détaillées.}
% 
